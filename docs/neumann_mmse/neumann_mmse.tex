\documentclass{article}
\usepackage{hyperref}
\usepackage{url}
\usepackage{graphicx}

\usepackage{caption}    % support subfigure
\usepackage{subcaption} % support subfigure
\usepackage{placeins}   % support FloatBarrier
\usepackage{listings}   % support codes
\usepackage{color}      % support color
\usepackage{amsmath} 	% support multiple lines
\usepackage{amsfonts,amssymb} % support holo characters
\usepackage{hyperref}	% hyper link
\usepackage[10pt]{extsizes}
%\documentstyle[nips14submit_09,times,art10]{article} % For LaTeX 2.09

% define color
\definecolor{darkred}{rgb}{0.6,0.0,0.0}
\definecolor{darkgreen}{rgb}{0,0.50,0}
\definecolor{lightblue}{rgb}{0.0,0.42,0.91}
\definecolor{orange}{rgb}{0.99,0.48,0.13}
\definecolor{grass}{rgb}{0.18,0.80,0.18}
\definecolor{pink}{rgb}{0.97,0.15,0.45}
% set the img folder
\graphicspath{{img/}}

% set the hyperlink
\hypersetup{hidelinks,
	colorlinks=true,
	allcolors=black,
	pdfstartview=Fit,
	breaklinks=true}


\renewcommand{\baselinestretch}{1.5}

\title{Threshold}
\author{}
\date{}

\usepackage{geometry}
\geometry{a4paper,left=2cm,right=2cm,top=2cm,bottom=2cm}

\begin{document}

\maketitle

\section{Neumann Series}
We review Neumann Series from \href{https://en.wikipedia.org/wiki/Neumann_series}{Wikipedia}. 
\begin{equation}
(I-T)^{-1} = \sum_{k=0}^\infty T^k
\end{equation}
\subsection{Approximate Matrix Inversion of LMMSE}
If we set $T = -T$, 
\begin{equation}
(I + T)^{-1} = \sum_{k=0}^\infty (-1)^kT^k.
\end{equation}
Let's look at linear MMSE equations
\begin{equation}
\begin{split}
\hat{x} &= (H^HH + \sigma^2I)^{-1}H^Hy\\
&= \frac{\sigma^2}{\sigma^2}(H^HH + \sigma^2I)^{-1}H^Hy\\
&= \frac{1}{\sigma^2}(\frac{H^HH}{\sigma^2} + I)^{-1}H^Hy\\
&= \frac{1}{\sigma^2} \sum_{k=0}^\infty (-1)^k(\frac{H^HH}{\sigma^2})^k H^Hy \\
&= \sum_{k=0}^\infty (-1)^k(\frac{1}{\sigma^2})^{k-1}(H^HH)^k H^Hy
\end{split}
\end{equation}

\end{document}
