\documentclass{article}

% general packages
\usepackage{url}
\usepackage{graphicx}
\usepackage{mathtools}  % math tools
\usepackage{caption}    % support subfigure
\usepackage{subcaption} % support subfigure
\usepackage{placeins}   % support FloatBarrier
\usepackage{listings}   % support codes
\usepackage{color}      % support color
\usepackage{amsmath} 	% support multiple lines
\usepackage{amsfonts,amssymb} % support holo characters
\usepackage[10pt]{extsizes}
%\documentstyle[nips14submit_09,times,art10]{article} % For LaTeX 2.09

%-----------------------------------------------
% special packages

\usepackage[numbers,sort&compress]{natbib}		% 文献引用连号时[1,2,3]变成[1-3]
\usepackage{fancyhdr} 					% 加载fancyhdr宏包
\usepackage[hidelinks]{hyperref}        % 支持超链接(隐藏超链接效果)
\usepackage{multirow} 					% 多行表格
\usepackage{float} 						% 图片禁止浮动
\usepackage{geometry}
%\usepackage{CJKutf8} 					% 支持中文——方案1(不支持中文页眉而放弃)
\usepackage[UTF8]{ctex}  				% 支持中文——方案2
\usepackage{makecell} 					% 单单元格调整
\usepackage{longtable} 					% 表格——跨页面
\usepackage{tabularx} 					% 表格——自动调整列宽
\usepackage{array} 						% 表格——用于 \newcolumntype
\usepackage{makecell} 					% 表格——独立表格
\usepackage[table]{xcolor}  			% 加载xcolor宏包,启用表格颜色功能
\usepackage{pifont} 					% 圆圈1-20: \ding{172} -\ding{181} 
\usepackage{algorithm, algpseudocode} 	% 伪代码
\usepackage{underscore} 				% 下划线
\usepackage{ulem} 					% 文字划线等

%-----------------------------------------------
% page settings
\geometry{a4paper,left=2cm,right=2cm,top=2cm,bottom=2cm}
\renewcommand{\baselinestretch}{1.2}
\setlength{\parindent}{0pt} % 设置段落缩进为 0
% 定义页眉风格
\pagestyle{fancy} 
\fancyhead[L]{} % 设置左页眉
\fancyhead[C]{Joint Channle Estimation \& Symbol Detection} % 设置中页眉
\fancyhead[R]{} % 设置右页眉
\renewcommand{\headrulewidth}{0.4pt} % 设置页眉线宽度(可选)


%-----------------------------------------------
% set math notations
\DeclarePairedDelimiter{\norm}{\lVert}{\rVert}
% paths 
\graphicspath{{../../img/}}
% define color
\definecolor{darkred}{rgb}{0.6,0.0,0.0}
\definecolor{darkgreen}{rgb}{0,0.50,0}
\definecolor{lightblue}{rgb}{0.0,0.42,0.91}
\definecolor{orange}{rgb}{0.99,0.48,0.13}
\definecolor{grass}{rgb}{0.18,0.80,0.18}
\definecolor{pink}{rgb}{0.97,0.15,0.45}
\definecolor{lightgreen}{RGB}{220, 255, 220}  % 浅绿色
\definecolor{lightred}{RGB}{255, 220, 220}    % 浅红色

%-----------------------------------------------
% 其他自定义设置
% 定义不同宽度的列类型
%\newcolumntype{TBX}{>{\hsize=1\hsize\centering\arraybackslash}X}
%\newcolumntype{MyCol}{>{\hsize=1.5\hsize\centering\arraybackslash}X}
%\newcolumntype{TBL}{>{\hsize=2\hsize\arraybackslash}X}
% 允许跨页
\allowdisplaybreaks[4]
% 设置标题编号深度,使其出现在目录中(如果需要)
\setcounter{secnumdepth}{4}
\setcounter{tocdepth}{4} % 如果你也希望它出现在目录中
% 伪代码
% 伪代码——跨行
\makeatletter
\newenvironment{breakablealgorithm}
  {% \begin{breakablealgorithm}
   \begin{center}
     \refstepcounter{algorithm}% New algorithm
     \hrule height.8pt depth0pt \kern2pt% \@fs@pre for \@fs@ruled
     \renewcommand{\caption}[2][\relax]{% Make a new \caption
       {\raggedright\textbf{\ALG@name~\thealgorithm} ##2\par}%
       \ifx\relax##1\relax % #1 is \relax
         \addcontentsline{loa}{algorithm}{\protect\numberline{\thealgorithm}##2}%
       \else % #1 is not \relax
         \addcontentsline{loa}{algorithm}{\protect\numberline{\thealgorithm}##1}%
       \fi
       \kern2pt\hrule\kern2pt
     }
  }{% \end{breakablealgorithm}
     \kern2pt\hrule\relax% \@fs@post for \@fs@ruled
   \end{center}
  }
\makeatother
% 伪代码——注释
\renewcommand{\algorithmiccomment}[1]{\hfill // #1}
\newcommand{\qxw}{Xinwei Qu}

% version
\newcommand{\version}{v1.0.2}

%-----------------------------------------------
% title
% 
\title{
	\Huge \textbf{Variational Bayes} \\[1em]
}
\author{\qxw}
\date{\today}

%-----------------------------------------------
% document
%
\begin{document}
% \begin{CJK}{UTF8}{gbsn} % 不支持中文页眉而放弃

\maketitle

\newpage
\tableofcontents
\newpage

\section{Variational Bayes}
For a telecommunication system, we have
\begin{equation}
y = Hx + z,
\end{equation}
where $y$ is the received signal, $x$ is the transmitted signal and $z\in \mathcal{CN}(0, \sigma^2)$. Please note that, 
\begin{equation}
x = x_p + x_d,
\end{equation}
where $x_p$ is the pilot and $x_d$ is data.
\subsection{Bayes Interference}
In the Rx, given the prior $p(y)$, we compute posterior distribution $p(y|x)$,
\begin{equation}
p(x|y) = \frac{p(x,y)}{p(y)} = \frac{\overbrace{p(y|x)}^{likelihood}\overbrace{p(x)}^{prior}}{\underbrace{p(y)}_{evidence}} = \frac{p(y|x)p(x)}{\int_y p(x,y)dy}
\end{equation}
Usually, we assume the evidence is 100\%, i.e., $p(y)=1$. Hence,
\begin{equation}
p(x|y) \propto \overbrace{p(y|x)}^{likelihood}\overbrace{p(x)}^{prior}
\end{equation}
Here, we need to choose \colorbox{yellow}{the likelihood and the prior}.
\subsection{Variational Interference}
However, the posterior may have no closed form, i.e., computing $p(x|y)$ is not feasible. Instead, we use a distribution $Q$ over the symbols $x$ to approximate $p(x|y)$, i.e.,
\begin{equation}
\begin{split}
q^*(x) &= \mathop{\arg\min}\limits_{q(x)\in Q} KL(q(x) || p(x|y)) \\
&= \mathop{\arg\min}\limits_{q(x)\in Q} \int_{x}q(x)\text{ln} \frac{q(x)}{p(x|y)}dx \\
&= \mathop{\arg\min}\limits_{q(x)\in Q} -\int_{x}q(x)\text{ln} \frac{p(x|y)}{q(x)}dx
\end{split}
\end{equation}
Here, $q^*(x)$ is the optimal $q(x)$ but \colorbox{yellow}{$p(x|y)$ is unknown}. Herefore,
\begin{align}
KL(q(x) || p(x|y)) &= -\int_{x}q(x)\text{ln} \frac{p(x|y)}{q(x)}dx \\
&= \int_{x}q(x)\text{ln} q(x)dx - \int_{x}q(x)\text{ln}  p(x|y)dx \\ 
&= \int_{x}q(x)\text{ln} q(x)dx - \int_{x}q(x)\text{ln}  \frac{p(x,y)}{p(y)}dx \\ 
&= \int_{x}q(x)\text{ln} q(x)dx - \int_{x}q(x)\text{ln}p(x,y)dx + \int_{x}q(x)\text{ln}p(y)dx \\ 
&= \int_{x}q(x)\text{ln} q(x)dx - \int_{x}q(x)\text{ln}p(x,y)dx + \text{ln}p(y)\int_{x}q(x)dx \\ 
&= \underbrace{\mathbb{E}_q [\text{ln} q(x)] - \mathbb{E}_q[\text{ln}p(x,y)]}_{-ELBO} + \text{ln}p(y) \\
&= -ELBO(q) + \text{ln}p(y)
\label{eq:vb-vi-kl}
\end{align}
\subsubsection{ELBO}
Here, ELBO is Evidence Lower Bound, i.e.,
\begin{align}
ELBO(q) &= \mathbb{E}_q[\text{ln}p(x,y)] - \mathbb{E}_q [\text{ln} q(x)] \label{eq:vb-vi-elbo}\\ 
&= \int_x q(x)\text{ln}\frac{\overbrace{p(x,y)}^{known}}{q(x)}dx \\
&= \int_x q(x)\text{ln}\frac{p(y|x)p(x)}{q(x)}dx
\end{align}
\eqref{eq:vb-vi-kl} can be rewritten as,
\begin{equation}
\begin{split}
\underbrace{\text{ln}p(y)}_{\text{CONST}} &= ELBO(q) + \underbrace{KL(q(x) || p(x|y))}_{\geq 0} \\
&\geq ELBO(q)
\end{split}
\end{equation}
The minimizing KL can be taken as the maximizing ELBO, i.e.,
\begin{equation}
\begin{split}
q^*(x) &= \mathop{\arg\min}\limits_{q(x)\in Q} KL(q(x) || p(x|y)) \\
&= \mathop{\arg\max}\limits_{q(x)\in Q} ELBO(q) \\
\end{split}
\end{equation}
\subsection{Mean Field}
Now, we know the problem has been simplified as the maximizing ELBO. Here, we use the mean-field assumption, i.e.,
\begin{equation}
\begin{split}
q(x) &= \prod_{i=1}^m q_i(x_i) \\
\text{ln}q(x) &= \sum_{i=1}^m \text{ln}q_i(x_i) \\
\mathbb{E}_q [\text{ln} q(x)] &=   \sum_{i=1}^m\mathbb{E}_{q_i} [\text{ln} q_i(x_i)] 
\end{split}
\label{eq:vb-mf}
\end{equation}
\subsubsection{Coordinate Ascent Optimization}
In $q = [q_1, q_2, \cdots, q_j, \cdots, q_m]$, we fix others to update $q_j$, i.e.,
\begin{equation}
\begin{split}
q_j^*(x_j) &= \mathop{\arg\min}\limits_{q_j} ELBO(q_j) \\
&= \frac{\text{exp}\{\mathbb{E}_{q_{-j}}[\text{ln}p(x,y)]\}}{\int_{x_j}\text{exp}\{\mathbb{E}_{q_{-j}}[\text{ln}p(x,y)]\}dx_j}
\end{split}
\label{eq:vb-mf-cao}
\end{equation}
To prove \eqref{eq:vb-mf-cao}, we need to load \eqref{eq:vb-mf} into \eqref{eq:vb-vi-elbo},
\begin{align}
ELBO(q) &= \mathbb{E}_q[\text{ln}p(x,y)] - \mathbb{E}_q [\text{ln} q(x)] \\
&= \int_{x}q(x)\text{ln}p(x,y)dx - \left[\mathbb{E}_{q_j} [\text{ln} q_j(x_j)] + \sum_{i\neq j}\mathbb{E}_{q_i} [\text{ln} q_i(x_i)]\right]
\label{eq:vb-mf-cao-pf-1}
\end{align}
Here, $\sum_{i\neq j}\mathbb{E}_{q_i} [\text{ln} q_i(x_i)]$ can be seen as a constant because it is not related to $q_j$. Therefore, \eqref{eq:vb-mf-cao-pf-1} can be simplified as ($*_{-j}$ represents the other elements except $j$),
\begin{align}
ELBO(q) &= \int_{x}q(x)\text{ln}p(x,y)dx - \mathbb{E}_{q_j} [\text{ln} q_j(x_j)] + \text{const}\\
&=\int_{x}q(x)\text{ln}p(x,y)dx - \int_{x_j}q(x_j)\text{ln} q_j(x_j)dx_j + \text{const}\\
&=\int_{x_j}\int_{x_{-j}}q(x_j)q(x_{-j})\text{ln}p(x,y)dx_jdx_{-j} - \int_{x_j}q(x_j)\text{ln} q_j(x_j)dx_j + \text{const} \\
&= \int_{x_j}q(x_j)\left[\int_{x_{-j}}q(x_{-j})\text{ln}p(x,y)dx_{-j}\right]dx_j - \int_{x_j}q(x_j)\text{ln} q_j(x_j)dx_j + \text{const} \\
&= \int_{x_j}q(x_j)\mathbb{E}_{q_{-j}}[\text{ln}p(x,y)]dx_j - \int_{x_j}q(x_j)\text{ln} q_j(x_j)dx_j + \text{const}
\label{eq:vb-mf-cao-pf-2}
\end{align}
\colorbox{yellow}{Here, we define a new distribution}
\begin{equation}
\begin{split}
\text{ln}\tilde{p_j}(x_j,y) &= \mathbb{E}_{q_{-j}}[\text{ln}p(x,y)] + const \\
\tilde{p_j}(x_j,y) &\propto \text{exp}\{\mathbb{E}_{q_{-j}}[\text{ln}p(x,y)]\}
\end{split}
\label{eq:vb-mf-cao-pf-dist}
\end{equation}
Here, we load \eqref{eq:vb-mf-cao-pf-dist} into \eqref{eq:vb-mf-cao-pf-2},
\begin{align}
ELBO(q) &= \int_{x_j}q(x_j)\text{ln}\tilde{p_j}(x_j,y)dx_j - \int_{x_j}q(x_j)\text{ln} q_j(x_j)dx_j + \text{const} \\
&= \int_{x_j}q(x_j)\text{ln}\frac{\tilde{p_j}(x_j,y)}{\text{ln} q_j(x_j)}dx_j+ \text{const} \\
&= -KL(q_j(x_j) || \tilde{p_j}(x_j,y))
\end{align}
The KL divergence reaches the minimun when 
\begin{equation}
\begin{split}
q_{x_j}^* &= \tilde{p_j}(x_j,y) \\
& \propto \text{exp}\{ \mathbb{E}_{q_{-j}}[\text{ln}p(x,y)] \} \\
&= \frac{\text{exp}\{ \mathbb{E}_{q_{-j}}[\text{ln}p(x,y)] \} }{\int_{x_j} \text{exp}\{ \mathbb{E}_{q_{-j}}[\text{ln}p(x,y)] \} dx_j}
\end{split}
\end{equation}
\colorbox{yellow}{$\int_{x_j} \text{exp}\{ \mathbb{E}_{q_{-j}}[\text{ln}p(x,y)] \} dx_j$是为了让总体概率为1}
\subsection{Algorithm Structure}
The structure is given as below:
\begin{breakablealgorithm}
	\begin{algorithmic}[1] %每行显示行号
		\State initialize $q_j(x_j)$ for $j\in{1,\cdots,m}$
		\While{ELBO not converge}
			\For{$j\in{1,\cdots,m}$}
				\State $q_{x_j}^*= \frac{\text{exp}\{ \mathbb{E}_{q_{-j}}[\text{ln}p(x,y)] \} }{\int_{x_j} \text{exp}\{ \mathbb{E}_{q_{-j}}[\text{ln}p(x,y)] \} dx_j}$
			\EndFor
			\State ELBO(q)= $\mathbb{E}_q[\text{ln}p(x,y)] - \mathbb{E}_q [\text{ln} q(x)] $
		\EndWhile
		\State \Return{$q(x)$}
	\end{algorithmic}
\end{breakablealgorithm}

\section{Variational Bayes in JCHESD}

\newpage
\bibliographystyle{IEEEtran}
\bibliography{../reference}
\end{document}