\documentclass{article}

\usepackage{textcase}
\usepackage{changes}
\usepackage{gensymb}
\usepackage{cite}
\usepackage{amsmath,amssymb,amsfonts}
\usepackage{algorithmic}
\usepackage{textcomp}
\usepackage{graphicx}
%\usepackage{citesort}
\usepackage{amsthm}			% 自定义环境样式包
\usepackage{amssymb}
\usepackage{amsfonts}
\usepackage{amsmath}
\usepackage{epsfig}
\usepackage{color}
\usepackage{fancybox}
\usepackage{textcomp}
\usepackage{multirow}
\usepackage{makecell}       % multiple lines inside a cell of a table
\usepackage{setspa ce}
\usepackage{psfrag}
\usepackage{booktabs}
\usepackage{float}
\usepackage{caption}
\usepackage{subcaption}
\usepackage{placeins}
%\floatstyle{plaintop}
\restylefloat{table}
\usepackage{tablefootnote}
%\usepackage{caption}
%\usepackage[caption = false]{subfig}
\usepackage[ruled]{algorithm2e}
\renewcommand{\algorithmcfname}{Algorithm}
\SetKwInput{KwData}{\textbf{Initialization}}
\usepackage{mathrsfs}
\newcommand{\blue}[1]{{\textcolor[rgb]{0,0,1}{#1}}}
\newcommand{\red}[1]{{\textcolor[rgb]{1,0,0}{#1}}}
\newtheorem{Lemma}{Lemma}
\newtheorem{Theorem}{Theorem}
\newtheorem{Remark}{Remark}
\newcommand{\CLASSINPUTtoptextmargin}{2.4cm}


%% 自定义包
\usepackage{bm}									%
\usepackage{enumitem}							% item支持更多格式
\usepackage[numbers,sort&compress]{natbib}		% 文献引用连号时[1,2,3]变成[1-3]
\usepackage{breqn}								% 公式过长自动换行

%% 自定义环境样式
\renewcommand{\proofname}{\textit{Proof:}}

% 覆盖 breqn 的编号设置
\renewcommand{\theequation}{\arabic{equation}}

% set the img folder
\graphicspath{{img/}}

%\documentstyle[nips14submit_09,times,art10]{article} % For LaTeX 2.09

% define color
\definecolor{darkred}{rgb}{0.6,0.0,0.0}
\definecolor{darkgreen}{rgb}{0,0.50,0}
\definecolor{lightblue}{rgb}{0.0,0.42,0.91}
\definecolor{orange}{rgb}{0.99,0.48,0.13}
\definecolor{grass}{rgb}{0.18,0.80,0.18}
\definecolor{pink}{rgb}{0.97,0.15,0.45}
% set the img folder
\graphicspath{{img/}}

\renewcommand{\baselinestretch}{1.5}

\title{OTFS Research Progress Report}
\author{Xinwei Qu}
\date{}

\usepackage{geometry}
\geometry{a4paper,left=2cm,right=2cm,top=2cm,bottom=2cm}

\begin{document}

\maketitle
\section{Introduction}
With the evolution from 5G to emerging 6G systems, Orthogonal Frequency Division Multiplexing (OFDM) has been the workhorse waveform in 4G/5G due to its ability to combat multipath using cyclic prefix and pilots \cite{10334727, 8647394}. However, its performance degrades significantly under severe Doppler in high-mobility scenarios, as inter-carrier interference (ICI) increases and pilot overhead rises \cite{8647394}.

Orthogonal Time Frequency Space (OTFS) was proposed as a next-generation modulation for 6G, operating in the delay–Doppler (DD) domain to transform a time-varying channel into an almost time-invariant one \cite{10334727, 8671740}. OTFS offers robustness to high mobility and enables stable channel interaction by localizing symbols in the DD domain \cite{10334727}.

To estimate the DD-domain channel, traditional OTFS schemes embed pilot symbols surrounded by a "guard" region, preventing pilot-data interference \cite{9539066}. However, guard symbols consume significant spectral resources, especially under large Doppler shifts where guards may occupy the entire Doppler axis, degrading spectrum efficiency markedly \cite{9536449}. For instance, works on superimposed pilot approaches explicitly aim to eliminate guard regions to recover spectral efficiency \cite{9536449}.

Existing multi-pilot and superimposed schemes alleviate guard overhead but introduce pilot-data interference, necessitating nontrivial joint estimation and detection mechanisms [5]–[7]. Moreover, these methods often assume fractional Doppler or pilot sparsity and employ compressive sensing, OMP, or message passing techniques \cite{9536449}.

Recent research has begun to explore "zero-guard" (0‑guard) approaches, fully eliminating guard overhead. However, in such systems pilots and data symbols inherently interfere, making separate channel estimation and symbol detection suboptimal [5], [7], [15]. Instead, a unified joint channel estimation and symbol detection framework is required to optimally extract both channel and transmitted data under interference.

Several promising joint frameworks have emerged:

Superimposed pilot + message passing: Mishra et al. propose a superimposed-pilot design with MMSE estimation and MP detection.

OAMP-based joint estimation/detection: A comprehensive survey proposes an OAMP + mean-field message passing method under no guard conditions.

Deep learning plugged-in frameworks: Zhang et al. design a PnP framework integrating DL denoisers for both channel estimation and symbol detection.

These methods focus on joint processing in high-Doppler channels. Nevertheless, they often include guard or pilot strategies that still incur inefficiencies, or employ learning approaches beyond traditional signal processing.

Motivation
To support high spectral efficiency in high-mobility 6G scenarios, a guard-free (“0‑guard”) OTFS design is ideal. However, without guard separation, pilots and data symbols collide in the DD domain, creating mutual interference. Conventional decoupled channel estimation and detection cannot address this interference effectively.

Proposed Approach
We advocate a joint channel estimation and symbol detection strategy tailored for zero-guard OTFS. This approach directly processes collided pilot and data symbols, jointly estimating the sparse DD-channel and detecting data symbols in an integrated optimization framework—without relying on guard regions or advanced coding techniques.

In the subsequent sections, we will detail:

The system model and problem formulation for zero-guard OTFS.

A joint sparse-estimation/detection algorithm based on OAMP or message passing.

Theoretical performance analysis under large Doppler and its spectral efficiency gains over guard-based schemes.

\subsection{Notations}
$a$, $\bm{a}$ and $\bm{A}$ demote scalar, vector, and matrix respectively. $\mathbb{C}^{M\times N}$ denotes the set of $M\times N$ dimensional complex matrix. $(\cdot)^T$, $(\cdot)^H$, $(\cdot)^*$, and $[\cdot]_M$ represent the transpose, Hermitian transpose, conjugate, and mod-$M$ operations.$\odot$ denotes Hadamard multiplication. $\text{diag}(\bm a)$ denotes the operation to diagonize a vector $\bm a$, $\text{off}(\bm{A})$ forces all diagonal elements ($A_{ii}$) to zero. We define $\bm{a}=vec(\bm{A})$ as the column-wise vectorization of matrix $\bm{A}$.

\section{System Model}
% todo
% T, L, \Delta f, K
% \sigma_d^2
% Xp, x_p, k_p, {l_p}_0, ..., {l_p}_last \sigma_d^2
We consider an OTFS system that transmits symbols $\bm{X_{DD}}\in \mathbb{C}^{K\times L}$ over the delay-Doppler (DD) domain grids $\{(k\Delta\tau, l\Delta v)|k=0,...,K-1, l=0,...,L-1\}$, where $\Delta\tau$ and $\Delta v$ denote the delay and Doppler resolutions, where:
\begin{itemize}[label={--}] % 用横线代替默认的圆点
    \item $\Delta\tau=T/L$ is the delay resolution, with $T$ being the symbol duration of the OTFS system,
    \item $\Delta v=\Delta f/$ is the Doppler resolution, with $\Delta f=1/T$ denoting the subcarrier spacing.
\end{itemize}

The system bandwidth is $L\Delta f$, and the OTFS frame duration is $NT$. The transmitter first maps the symbols $x[k,l]$ to the time-frequency (TF) domain grids $\{(l\Delta f, kT)|\}$ via the inverse finite symplectic Fourier transform (ISFFT). The time-domain signal is synthesized using a conventional OFDM modulator with a transmit shaping pulse yet employs a single initial cyclic prefix spanning the full delay-Doppler spread duration, contrasting with conventional OFDM's per-symbol cyclic prefix insertion. The time-domain signal is transmitted over a time-varying wireless channel characterized by the delay-Doppler impulse response $h(\tau, v)$ as \cite{7925924},
\begin{equation}
h(\tau, v) = \sum_{i=1}^P h_i \delta (\tau - {\tau}_i) \delta (v - v_i) ,
\end{equation}
where $\delta(\cdot)$ denotes the Dirac delta function, $h_i \sim \mathcal{N}(0, \frac{1}{P})$ is the gain of the $i$-th propagation path, and $P$ represents the total number of paths. Each path is characterized by distinct delay and/or Doppler shifts, modeling the channel response between the receiver and either moving reflectors or the transmitting source. The delay and Doppler shifts are given as,
\begin{equation}
{\tau}_i = l_i \frac{T}{L}, v_i = k_i \frac{\Delta f}{K},
\end{equation}
respectively. Let the integers $l_i \in [0, l_{max}]$ and $k_i \in [-k_{max}, k_{max}]$ represent the delay and Doppler shift indices, respectively, where $l_{max}$ and $k_{max}$ denote the maximum delay index and maximum Doppler shift index across all propagation paths. Note that we restrict our consideration to integer-valued indices, as fractional delay and Doppler shifts can be equivalently represented through virtual integer taps in the delay-Doppler domain using the techniques described in \cite{6563167, 8377159, 8516353}.

At the receiver, the time-domain signal is first converted to the time-frequency (TF) domain through matched filtering and OFDM demodulation, then transformed to the delay-Doppler (DD) domain via inverse symplectic finite Fourier transform (ISFFT), yielding the received symbols $\bm{Y}_{\text{DD}}\in \mathbb{C}^{K\times L}$. The DD domain input-output relationship can be formulated in vector form as \cite{10264119},
\begin{equation}
\bm{y} = \bm{Hx} + \bm{\tilde{z}},
\label{eq:sys-DD}
\end{equation}
where $\bm{x} = vec(\bm{X_{DD}}^T)$, $\bm{y} = vec(\bm{Y_{DD}}^T)$, and $\tilde{z} \sim \mathcal{CN}(0, \sigma_z^2)$ is an independent and identically distributed (i.i.d.) Gaussian noise \cite{8516353, 10264119, 7925924}.

%% todo
% 解释 H 在两种Pulse的值
\section{System Model}
% todo
% T, L, \Delta f, K
% \sigma_d^2
% Xp, x_p, k_p, {l_p}_0, ..., {l_p}_last \sigma_d^2
We consider an OTFS system that transmits symbols $\bm{X_{DD}}\in \mathbb{C}^{K\times L}$ over the delay-Doppler (DD) domain grids $\{(k\Delta\tau, l\Delta v)|k=0,...,K-1, l=0,...,L-1\}$, where $\Delta\tau$ and $\Delta v$ denote the delay and Doppler resolutions, where:
\begin{itemize}[label={--}] % 用横线代替默认的圆点
    \item $\Delta\tau=T/L$ is the delay resolution, with $T$ being the symbol duration of the OTFS system,
    \item $\Delta v=\Delta f/$ is the Doppler resolution, with $\Delta f=1/T$ denoting the subcarrier spacing.
\end{itemize}

The system bandwidth is $L\Delta f$, and the OTFS frame duration is $NT$. The transmitter first maps the symbols $x[k,l]$ to the time-frequency (TF) domain grids $\{(l\Delta f, kT)|\}$ via the inverse finite symplectic Fourier transform (ISFFT). The time-domain signal is synthesized using a conventional OFDM modulator with a transmit shaping pulse yet employs a single initial cyclic prefix spanning the full delay-Doppler spread duration, contrasting with conventional OFDM's per-symbol cyclic prefix insertion. The time-domain signal is transmitted over a time-varying wireless channel characterized by the delay-Doppler impulse response $h(\tau, v)$ as \cite{7925924},
\begin{equation}
h(\tau, v) = \sum_{i=1}^P h_i \delta (\tau - {\tau}_i) \delta (v - v_i) ,
\end{equation}
where $\delta(\cdot)$ denotes the Dirac delta function, $h_i \sim \mathcal{N}(0, \frac{1}{P})$ is the gain of the $i$-th propagation path, and $P$ represents the total number of paths. Each path is characterized by distinct delay and/or Doppler shifts, modeling the channel response between the receiver and either moving reflectors or the transmitting source. The delay and Doppler shifts are given as,
\begin{equation}
{\tau}_i = l_i \frac{T}{L}, v_i = k_i \frac{\Delta f}{K},
\end{equation}
respectively. Let the integers $l_i \in [0, l_{max}]$ and $k_i \in [-k_{max}, k_{max}]$ represent the delay and Doppler shift indices, respectively, where $l_{max}$ and $k_{max}$ denote the maximum delay index and maximum Doppler shift index across all propagation paths. Note that we restrict our consideration to integer-valued indices, as fractional delay and Doppler shifts can be equivalently represented through virtual integer taps in the delay-Doppler domain using the techniques described in \cite{6563167, 8377159, 8516353}.

At the receiver, the time-domain signal is first converted to the time-frequency (TF) domain through matched filtering and OFDM demodulation, then transformed to the delay-Doppler (DD) domain via inverse symplectic finite Fourier transform (ISFFT), yielding the received symbols $\bm{Y}_{\text{DD}}\in \mathbb{C}^{K\times L}$. The DD domain input-output relationship can be formulated in vector form as \cite{10264119},
\begin{equation}
\bm{y} = \bm{Hx} + \bm{\tilde{z}},
\label{eq:sys-DD}
\end{equation}
where $\bm{x} = vec(\bm{X_{DD}}^T)$, $\bm{y} = vec(\bm{Y_{DD}}^T)$, and $\tilde{z} \sim \mathcal{CN}(0, \sigma_z^2)$ is an independent and identically distributed (i.i.d.) Gaussian noise \cite{8516353, 10264119, 7925924}.
\section{Methodology}

To solve the problem of joint channel estimation and symbol detection in a zero-guard OTFS system, we propose a novel framework that leverages Graph Neural Networks (GNNs) and attention mechanisms.

First, we model the OTFS delay-Doppler grid as a graph, where each vertex corresponds to a symbol node (either pilot or data), and edges capture neighborhood correlations induced by the underlying sparse DD-domain channel~\cite{Kipf2017_GCN}. The edge weights are initialized based on expected sparsity patterns and refined during training.

We adopt a GNN to propagate information across the graph, enabling local estimation at each node to incorporate structural dependencies. To focus learning on more informative neighbors, we further integrate a multi-head attention mechanism into the GNN architecture~\cite{Velickovic2018_GAT}. The attention scores are dynamically learned and reflect the relative importance of neighboring nodes in estimating each symbol's likelihood and the local channel response.

Our joint estimator performs the following iterative steps:
\begin{enumerate}
    \item Message passing via GNN layers to update node features.
    \item Attention-weighted aggregation to highlight relevant features.
    \item Joint estimation of channel and symbol values using an MSE loss between predicted and received values.
\end{enumerate}

This architecture is trained end-to-end using labeled synthetic OTFS data under a wide range of Doppler spreads and SNRs. Compared with classical separate estimation or conventional model-based joint detection, our framework achieves lower BER and improved channel estimation accuracy in high-mobility scenarios.


\bibliographystyle{IEEEtran}
\bibliography{myBib}

\end{document}
